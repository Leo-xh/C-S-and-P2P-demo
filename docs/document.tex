\documentclass[15pt]{ctexart}
\usepackage{xeCJK}
	\setCJKmainfont{SimSun}         % 缺省中文字体为宋体
	\setmainfont{Times New Roman}   % 缺省英文字体 Times New Roman
\usepackage{geometry}
	% \geometry{left=2.5cm,right=2.5cm,top=2.5cm,bottom=3.5cm}
	\geometry{bottom=3.5cm}
\usepackage{float}
\usepackage{listings}
\usepackage{graphicx}
\usepackage{appendix}
\usepackage[colorlinks]{hyperref}
\usepackage{amsmath}
\usepackage{indentfirst}
\usepackage{enumerate}
\usepackage{fancyhdr}
\usepackage{textcomp}
\usepackage{appendix} 
\usepackage{multirow}
\usepackage{geometry}
\geometry{verbose,letterpaper}
\usepackage{media9}
\pagestyle{fancy}
\lhead{应用层——C/S与P2P通信}
\rhead{\thepage}
\cfoot{}
\lstset{frameshape={RYRYNYYYY}{yny}{yny}{RYRYNYYYY}, backgroundcolor=\color[RGB]{245,245,244}}

\begin{document}
\input{cover}
\tableofcontents
\newpage
\section{项目介绍} % (fold)
\label{sec:项目介绍}
\par 这是一个应用层的通信应用项目,包括一个服务器-客户端模型和P2P模型,两者的功能都是传输文件,项目主页是\url{https://github.com/Leo-xh/C-S-and-P2P-demo}。其中,服务器-客户端模型使用的是单服务器多客户端模型,并且单一客户端可以同时请求多个文件,服务器和客户端都使用多线程模型。P2P模型参考的是bittorrent协议,完成了bittorrent协议的一个实现(命名为Compact Bittorrent Protocol/1.0),并且实现了原来的biitorrent协议中的几个扩展协议。
	\subsection{设计应用层协议的原则} % (fold)
	\label{sub:设计应用层协议的原则}
		一个应用层协议应当包含两个主要部分:
		\begin{enumerate}
			\item 编码控制(编码和译码)。
			\item 流程控制。
		\end{enumerate}
		\par 一个协议可以根据多种原则来进行划分,比如按照编码分类、按照协议边界分类。而对于协议的评判来讲,主要有四点评判原则:
		\begin{enumerate}
			\item 协议的高效性,包括打包和解包的效率、数据压缩率等等。关于这一点,应用层协议的效率主要涉及数据方面,因为数据的传输并不是应用层的责任。
			\item 协议的简单性。
			\item 协议的可扩展性。举个例子,在Bittorrent协议中,有许多扩展,本项目中也实现了几个扩展,这些扩展使得Bittorrent更加健壮和高效。
			\item 协议应当向前和向后兼容。拿上面的Bittorrent扩展为例,实现了这些扩展的协议版本兼容了为实现这些扩展的协议版本。
		\end{enumerate}
		\par 设计协议时,主要需要回答三个基本问题:
		\begin{enumerate}
			\item 这个协议需要完成什么问题。
			\item 协议中传输的信息的含义是什么。
			\item 该协议的特征是什么。
		\end{enumerate}
		下面本项目中设计的协议都将围绕这三个问题。
	% subsection 设计应用层协议的原则 (end)
% section 项目介绍 (end)
\section{C/S通信} % (fold)
\label{sec:c_s通信}
\par 本项目中实现的C/S通信模型使用python实现,主要利用的是socket,threading,struct,os等常用库,其中服务器使用多线程,能够支持多个客户端同时请求文件;客户端也使用多线程,能够同时请求多个文件。
\par 提供的服务如下:
\begin{enumerate}[1.]
	\item 原始数据传输。
	\item 加密数据传输。
	\item 查看服务器的文件目录。
\end{enumerate}
\subsection{协议设计} % (fold)
\label{sub:协议设计}
\par 这是一个二进制模糊边界和固定边界的协议,即传输的数据以二进制编码,在请求报文中能够确定报文长度,在应答报文中无法明确知道协议报文的长度,需要通过报文中的长度字段知道。
\par 在设计协议格式时,经过小组的讨论,决定在协议头中采用两种(请求和发文件)报文格式,请求报文头中有:类型、服务、版本、序号和文件名字段,而发文件报文头中有:类型、服务、版本、序号、长度、错误码和数据体;
每个字段的作用在后面有相应解释。定义好协议格式后,接着定义客户端和服务器之间的行为交互:首先客户端在链接上服务器时,1、可以向服务器发出“查询文件目录”请求,
然后服务器查询本地的文件及目录,以明文形式发回去给客户端,客户端拿到后输出回显到屏幕上;2、客户端发出“某文件”请求,并选择是否需要加密,服务器接收请求,若文件不存在,发送一个错误码为1的报文
给客户端,客户端根据该报文做出响应;若文件存在,服务器将文件分成许多报文向客户端发送(根据加密需要选择加密与否),并且服务器在自己的终端上显示发送文件时的进度条,
等文件的所有报文发送完成,服务器再向客户端发送一个数据长度为0的报文,告知客户端,文件发送完毕。
\par 接着分析了该应用层协议是用来传输二进制文件(主要功能)和传输明文(次要功能),对于明文来说,就不得不考虑编码转换的问题,
因为需要采用了struct库的pack和unpack函数来进行压包和解包的。而且,在明确了该应用层协议是采用TCP之后,我们进行了下面的TCP缺陷分析:
\par 在实现这个通信模型时会遇到的问题主要是TCP协议的分包和粘包问题,TCP中只有数据流这样的概念,而没有数据包这一类的概念;
因为自己设计应用层调用网络编程的API接口,一次调用中要发送的报文大小在API接口中没有规定,但是在机器的IP层,TCP并不是按照
调用者传给API的报文大小一次性发出(我们已经知道在以太网中TCP报文大小限制在1500bytes),而是采用自己的算法,将包切割,
然后发出,包到达对方的IP层时再进行重组(TCP提供了可靠的服务,不需要考虑丢包问题),但是由于IP层的算法设计,导致接收方的IP层
重组后的传给应用层的报文,并不一定完全等于发送方的应用层传给IP层的报文,会可能出现:分包(发送方应用层的一个报文,实际上分成了两个接收方应用层的报文)
和粘包(发送方应用层一个报文的部分(或者全部)与另一个报文的部分(或者全部),实际上合成接收方应用层的一个报文)。
每次收到的不一定会是一个完整的数据包,所以需要通过某种方法明确当前处理的数据包的大小,然后解析这个大小的数据包。
\par 这类问题的解决方法大致有两种:第一种是,设计一个独特的分隔符,将应用层的每个包后面加上该分隔符,然后接收方根据分隔符来
分割出每个包,但是因为我们这次传输的是文件(二进制),无法预测可以采用什么充当分隔符而不会在文件上出现;第二种是:发送方在报文协议头上,
加上一个字段,表明该报文数据字段的长度,然后接收方的应用层设置一个报文缓冲区,每个到达的报文存在缓冲区中,
首先在缓冲区中提取报文头(固定长度的),然后根据报文头的长度字段再在缓冲区中提取相应长度的数据字段(当缓冲区现有数据长度小于需要提取的长度时,继续收包,
直到长度大于等于需要提取的长度),这样便能完全解决由于TCP数据流的特点而产生的缺陷问题。我们本次协议采用的是第二种方案。
\par 至于在加密过程中,我们采用的是AES对称加密算法,原因是:对称加密,适合数据的加密和加密;而且实现较为简单。

\par 客户端的请求报文设计如表\ref{tab:client}所示。
\begin{table}[H]
	\centering
	\begin{tabular}{|c|c|c|c|}
		\hline
		类型 & 服务 & 版本 & 序号 \\
		\hline
		\multicolumn{4}{|c|}{\multirow{6}*{文件名}
		}
		\\
		\multicolumn{4}{|c|}{~}\\
		\multicolumn{4}{|c|}{~}\\
		\multicolumn{4}{|c|}{~}\\
		\multicolumn{4}{|c|}{~}\\
		\multicolumn{4}{|c|}{~}\\
		\multicolumn{4}{|c|}{~}\\
		\multicolumn{4}{|c|}{~}\\
		\hline
	\end{tabular}
	\caption{客户端请求报文}
	\label{tab:client}
\end{table}
\par 参数解释如下:类型指的是协议号,服务是服务号(3种不同的服务),版本是协议版本号,序号是请求序号,大小都是2字节,文件名指的是请求的文件名,长度上限为200字节。
\par 服务器响应报文设计如表\ref{tab:server}:
\begin{table}[H]
	\centering
	\begin{tabular}{|c|c|c|c|c|c|}
		\hline
		类型 & 服务 & 版本 & 序号 & 长度 & 错误码 \\
		\hline
		\multicolumn{6}{|c|}{\multirow{6}*{数据}}
		\\
		\multicolumn{6}{|c|}{~}\\
		\multicolumn{6}{|c|}{~}\\
		\multicolumn{6}{|c|}{~}\\
		\multicolumn{6}{|c|}{~}\\
		\multicolumn{6}{|c|}{~}\\
		\multicolumn{6}{|c|}{~}\\
		\multicolumn{6}{|c|}{~}\\
		\hline
	\end{tabular}
	\caption{服务器应答报文}
	\label{tab:server}
\end{table}
\par 类型、服务、版本、序号字段和请求报文中一样,长度字段指的是数据字段的长度,大小为2字节,数据字段是发送往客户端的数据。
\par 对于上面提到的三种服务,对应的服务号分别是0,1,2。
\par 而错误码字段现在只有两种:0表示服务器有客户端需要的文件;1表示服务器没有客户端需要的文件。
\par 服务器和客户端的控制流程如图\ref{fig:server}和\ref{fig:client}所示。
\begin{figure}[H]
	\begin{minipage}{0.5\linewidth}
		\flushleft
		\includegraphics[width=1.1\linewidth]{imgs/server.png}
		\caption{服务器流程图}
		\label{fig:server}
	\end{minipage}
	\begin{minipage}{0.45\linewidth}
		\flushright
		\includegraphics[width=1\linewidth]{imgs/client.png}
		\caption{客户端流程图}
		\label{fig:client}
	\end{minipage}
\end{figure}
% subsection 协议设计 (end)
\subsection{特点} % (fold)
\label{sub:特点}
本项目中实现的C/S模型的特点是:
\begin{enumerate}
	\item 实现了支持多文件同时传输文件时的传输进度条,即多个进度条能够同时正常显示。
	\item 实现了服务器的文件查找功能。
	\item 使用了数据加密技术。
	\item 客户端能够同时请求多个文件,服务器能够同时处理多个请求。
\end{enumerate}
% subsection 特点 (end)

% section c_s通信 (end)

\section{P2P通信} % (fold)
\label{sec:p2p通信}
	本项目中完成了bittorrent协议的一个实现(命名为Compact Bittorrent Protocol/1.0),这个实现采用了基本的Bittorrent协议规定的内容,同时参考了社区中的建议(比如Block大小的选择)。同时我们参考了Bittorrent协议网站上列出的的扩展方案中的协议。我们的实现主要包括两个方面,一是客户端(Peer在和Tracker通信时作为客户端)与服务器(Tracker)的通信;二是对等方之间的通信。下面将围绕这两个方面进行介绍。
	首先一个Peer的程序流程和状态转移如图\ref{fig:peer}所示。
	\begin{figure}[H]
		\centering
		\includegraphics[scale=0.3]{imgs/peer.png}
		\caption{Peer程序流程和状态图}
		\label{fig:peer}
	\end{figure}
	\subsection{Twisted和异步编程} % (fold)
	\label{sub:twisted和异步编程}
		\subsubsection{Twisted简介} % (fold)
		\label{ssub:twisted简介}
			\par Twisted是用Python实现的基于事件驱动的网络引擎框架,Twisted支持许多常见的传输及应用层协议,包括TCP、UDP、SSL/TLS、HTTP、IMAP、SSH、IRC以及FTP。
			\par 在本项目中主要使用的是以下几个类:
			\begin{enumerate}
				\item Transport:Transports代表网络中两个通信结点之间的连接。Transports负责描述连接的细节,比如连接是面向流式的还是面向数据报的,流控以及可靠性。
				\item Protocol:Protocols描述了如何以异步的方式处理网络中的事件。
				\item Reactor:Twisted实现了设计模式中的反应堆(reactor)模式,这种模式在单线程环境中调度多个事件源产生的事件到它们各自的事件处理例程中去。Twisted的核心就是reactor事件循环。Reactor可以感知网络、文件系统以及定时器事件。它等待然后处理这些事件,从特定于平台的行为中抽象出来,并提供统一的接口,使得在网络协议栈的任何位置对事件做出响应都变得简单。
				\item Deferred:Deferred对象以抽象化的方式表达了一种思想,即结果还尚不存在。它同样能够帮助管理产生这个结果所需要的回调链。当从函数中返回时,Deferred对象承诺在某个时刻函数将产生一个结果。返回的Deferred对象中包含所有注册到事件上的回调引用,因此在函数间只需要传递这一个对象即可,跟踪这个对象比单独管理所有的回调要简单的多。
			\end{enumerate}
		% subsubsection twisted简介 (end)
		\subsubsection{异步编程与事件驱动} % (fold)
		\label{ssub:异步编程与事件驱动}
			图是一张区分单线程、多线程、和异步编程模型的示意图。
		\begin{figure}[H]
			\centering
			\includegraphics[scale=0.4]{imgs/threading_models.png}
			\caption{示意图}
			\label{fig:threading_models}
		\end{figure}
			\par 使用异步编程的原因在于网络应用程序通常具有多任务高度独立、某些事件的发生会阻塞另外的事件的特点。
		% subsubsection 异步编程与事件驱动 (end)

	% subsection twisted和异步编程 (end)
	\subsection{Tracker和Client的通信协议(UDP Tracker Protocol)} % (fold)
	\label{sub:tracker_和_client}
		\subsubsection{协议设计} % (fold)
		\label{ssub:协议设计}
			\par tracker服务器需要能够维护一个发起种子文件链接请求的对等方列表,将延时过期的对等方从对等方列表中剔除,
			并响应新的对等方的请求,以及向所有对等方发送实时的对等方列表。
			\par 客户端是每个对等方与tracker服务器链接的功能部分,
			向服务器发起种子文件链接请求,拿到tracker服务器返回来的对等方列表,交给本地的对等方,并且需要根据服务器设定的时间要求,
			一定时间再次发起一个请求,告知服务器该对等方还存活以及拿到新对等方列表并更新本地的对等方列表。
			\\
			\par tracker服务器与客户端之间传输的报文包括以下四种:
			\begin{table}[H]
				\centering
				\begin{tabular}{|c|c|c|c|}
					\hline
					protocol\_id & action & transaction\_id \\
					\hline
				\end{tabular}
				\caption{客户端connect request}
			\end{table}
			\par 参数及作用:action是表明报文类型(connect还是announce),
			transaction\_id是客户端生成并以此来认证tracker返回的报文。
			\begin{table}[H]
				\centering
				\begin{tabular}{|c|c|c|c|}
					\hline
					action & transaction\_id & connection\_id \\
					\hline
				\end{tabular}
				\caption{服务器connect response}
			\end{table}
			\par 参数及作用:connecttion\_id是tracker生成了独一无二的一个验证码,用来验证客户端请求的合法性,
			并防止遭受大量无效的请求。
			\begin{table}[H]
				\centering
				\begin{tabular}{|c|c|c|c|}
					\hline
					connection\_id & action     & transaction\_id & info\_hash \\
					\hline
					peer\_id       & downloaded & left            & uploaded   \\
					\hline
					event          & IP address & key             & num\_want  \\
					\hline
					port           &            &                 &            \\
					\hline
				\end{tabular}
				\caption{客户端announce request}
			\end{table}
			\par 参数及作用:IP address是可选选项,port是客户端对应的对等方监听的端口。
			
			\begin{table}[H]
				\centering
				\begin{tabular}{|c|c|c|}
					\hline
					action & transaction\_id & interval \\
					\hline
					\multicolumn{3}{|c|}{\multirow{5}*{peer\_list}
					}\\
					\multicolumn{3}{|c|}{~}\\
					\multicolumn{3}{|c|}{~}\\
					\multicolumn{3}{|c|}{~}\\
					\multicolumn{3}{|c|}{~}\\
					\hline
				\end{tabular}
				\caption{服务器announce response}
			\end{table}
			\par 参数及作用:interval是服务器自定的时间间隔参数,要求client在该时间间隔后必须再次向tracker发出一次announce request,
			若tracker在两倍的interval时间后,没有收到client发来的announce request,则认为client已失去链接,将它从peer\_list中剔除出去。
			\\
			\par 在tracker和client通信中,为了减轻tracker负担,我们选择了UDP,而在采用UDP协议时,就决定了应用层需要去解决丢包、超时的情况,
			所以在client中需要设计超时重传机制(而不是在tracker服务器中),而且每发生了超时重传后,下一次的的超时重传间隔设得比现在的间隔要大。
		% subsubsection 协议设计 (end)
		\subsubsection{协议特点} % (fold)
		\label{ssub:协议特点}
			tracker和client通信设计的特点是:
			\begin{enumerate}
				\item tracker使用twisted事件驱动型编程框架来实现,以client的请求这一事件来驱动tracker的响应,极大程度上降低tracker在没有client请求时的资源消耗。
				\item 采用UDP协议,并为client设计了较为完备超时重传的机制,以此有效降低tracker的服务负载。
				\item tracker给client发送的peer\_list采用的是紧凑格式的,每个peer的信息只需要用到6字节,相比于采用bencoded字典的列表,数据的压缩率达到了50\%以上的提高,			
			\end{enumerate}		
		% subsubsection 协议特点 (end)
	% subsection tracker_和_client (end)
	\subsection{Peer协议(Peer Protocol)} % (fold)
	\label{sub:peer}
		如图\ref{fig:peer}所示,peer协议主要处理了以下问题:
		\begin{enumerate}
			\item 判断多个peer是否在同一个torrent中。
			\item peer间请求文件和接受文件,其中涉及选择策略和各个peer之间信息交换(主要是peer拥有哪些piece)。
			\item 处理文件写入和断点续传。
		\end{enumerate}
		\subsubsection{协议设计} % (fold)
		\label{ssub:协议设计}
			\noindent \textbf{Peer间通信}\\
			\begin{enumerate}
				\item \emph{握手}。
				一个Peer会在一定时间间隔后定时连接没有连接过的peer,接受握手后会检测是否应当保持连接。
				\item \emph{请求和发送文件}。
				使用一定的策略进行piece和peer的选择。
				\item \emph{报文设计}。
				主要有两种报文,一是handshake报文,二是message报文。前者用以握手,后者用来进行各种通信。
			\end{enumerate}
		% subsubsection 协议设计 (end)
		\subsubsection{协议特点} % (fold)
		\label{ssub:协议特点}
			\begin{enumerate}
				\item 支持断点续传。
			\end{enumerate}
		% subsubsection 协议特点 (end)
	
	% subsection peer通信 (end)

% section p2p通信 (end)

\section{安装和部署} % (fold)
\label{sec:安装和部署}

% section 安装和部署 (end)

\section{结果} % (fold)
\label{sec:结果}
\subsection{结果展示} % (fold)
\label{sub:结果展示}
首先介绍本项目测试时的网络拓扑图,如图\ref{fig:NetworkCS}和\ref{fig:NetworkP2P}所示。
\begin{figure}[H]
\begin{minipage}{0.48\linewidth}
	\centering
	\includegraphics[width=0.9\linewidth]{imgs/NetworkCS.png}
	\caption{C/S模型网络拓扑图}
	\label{fig:NetworkCS}
	\end{minipage}
	\centering
	\begin{minipage}{0.48\linewidth}
	\includegraphics[width=0.9\linewidth]{imgs/NetworkP2P.png}
	\caption{P2P模型网络拓扑图}
	\label{fig:NetworkP2P}
	\end{minipage}
\end{figure}
\\
\\
\\
\\
\\
\\
	C/S模式结果展示。
\\
\\
\\
\\
\\
\\
\\
\\
\\
\\
\\
\\
\\
\\
	P2P模式结果展示。
\\
\\
\\
\\
\\
\\
\\
\\
\\
\\
\\
\\
\\
\\
% subsection 结果展示 (end)
\subsection{对比} % (fold)
\label{sub:对比}
	使用P2P模式(简化Bittorrent协议)在Peers数目较少时要比使用C/S模式要慢上许多,而在Peers数目较多时速度有较大提升。
% subsection 对比 (end)

% section 结果 (end)
\section{总结} % (fold)
\label{sec:总结}
	协议的设计是相当复杂的,主要在于协议报文主体的设计和协议控制流程的设计。
% section 总结 (end)

\section{项目管理记录} % (fold)
\label{sec:项目管理记录}
	本项目主要使用的是github来管理项目,建立了一个仓库,三人同时使用。
% section 项目管理记录 (end)

\newpage
\appendixpage
\begin{appendices}
	\section{参考文献} % (fold)
	\begin{enumerate}
		\item Jonas Fonseca,et al,\url{http://jonas.nitro.dk/bittorrent/bittorrent-rfc.html#anchor17},Bittorrent协议详细解读。
		\item Bram Cohen,\url{http://www.bittorrent.org/beps/bep_0003.html},Bittorrent官方协议。
		\item Jessica McKellar & Abe Fettig,Twisted Network Programming Essentials(second edition)。
	\end{enumerate}
	% section  (end)
\end{appendices}

\end{document}